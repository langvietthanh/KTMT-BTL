\subsection*{\textbf{\Large BÀI 2: }}
\addcontentsline{toc}{subsection}{BÀI 2}

\subsubsection*{1. Khảo sát cấu hình của máy và hệ thống bộ nhớ của máy đang sử dụng (Bộ nhớ trong: ROM, RAM, Cache System, Bộ nhớ ngoài: ổ đĩa
cứng, CD, Thiết bị vào ra.)}
\addcontentsline{toc}{subsubsection}{Khảo sát cấu hình và hệ thống bộ nhớ của máy}


\textbf{Bộ vi xử lí (CPU)}
\begin{figure}[H]
    \centering
    \includegraphics[width=0.55\linewidth]{cau hinh may.png}
    \caption{CPU}
    \label{fig:cau_hinh_may}
\end{figure}

\begin{itemize}
    \item CPU: Intel Core i7-1165G7
    \item Thế hệ: 11th Gen Tiger Lake-U
    \item Socket: 1449 FCBGA
    \item Công nghệ sản xuất: 10nm
    \item Số nhân / Số luồng: 4 Cores / 8 Threads
    \item Xung nhịp cơ bản: 2.80 GHz
    \item Xung nhịp thực tế: 2494.15 MHz (2.49 GHz tại thời điểm khảo sát)  
\end{itemize}
%============================================================================
\textbf{Bộ nhớ đệm (Cache)}
\begin{figure}[H]
    \centering
    \includegraphics[width=0.55\linewidth]{cache.png}
    \caption{Cache}
    \label{fig:cache}
\end{figure}

\begin{itemize}
    \item L1: 4 x 48 KB (Data) + 4 x 32 KB (Instruction)
    \item L2: 4 x 1.25 MB
    \item L3: 12 MB
\end{itemize}
%============================================================================
\textbf{Bộ nhớ trong (RAM)}
\begin{figure}[H]
    \centering
    \begin{subfigure}[h]{0.45\textwidth}
        \centering
        \includegraphics[width=\linewidth]{memory.png}
        \caption{Memory}
        \label{fig:computer_memory}
    \end{subfigure}
    \begin{subfigure}[h]{0.45\textwidth}
        \centering
        \includegraphics[width=\linewidth]{spd.png}
        \caption{SPD}
        \label{fig:computer_spd}
    \end{subfigure}
    \caption{Cấu hình bộ nhớ trong (RAM)}
    \label{fig:memory}
\end{figure}

\begin{itemize}
    \item Dung lượng tổng: 16 GB DDR4
    \item Tốc độ DRAM: 1597.1 MHz (tức là ~3200 MHz thực tế do DDR: Double Data Rate)
    \item Hãng sản xuất: Ramaxel Technology
    \item Cấu hình: 2 kênh (Dual Channel - 2x 64-bit)
    \item Thông số Timing:
        \begin{itemize}
        \item CAS Latency (CL): 22
        \item RAS to CAS Delay (tRCD): 22
        \item RAS Precharge (tRP): 22
        \item Cycle Time (tRAS): 52
        \end{itemize}
\end{itemize}
%============================================================================
\textbf{Bộ nhớ ngoài (ROM)}
\begin{figure}[H]
    \centering
    \includegraphics[width=0.8\linewidth]{task_manager.png}
    \caption{ROM}   
    \label{fig:computer_rom}
\end{figure}

\begin{itemize}
    \item Ổ cứng chính: SSD NVMe
    \item Model: SK Hynix HFS512GEJ4X112N
    \item Dung lượng: 512 GB (hiển thị 477 GB khả dụng)
    \item Tốc độ phản hồi trung bình: 1.6 ms
    \item Giao tiếp: NVMe (rất nhanh so với SATA SSD)
\end{itemize}

%============================================================================

\textbf{Thiết bị vào/ra}
\begin{figure}[H]
    \centering
    \includegraphics[width=\linewidth]{divice io.png}
    \caption{Divice I/O}
    \label{fig:computer_divice}
\end{figure}

%============================================================================

\subsubsection*{2. Dùng công cụ Debug khảo sát nội dung các thanh ghi IP, DS, ES, SS, CS, BP, SP}
\addcontentsline{toc}{subsubsection}{Dùng công cụ Debug khảo sát nội dung các thanh ghi}

\begin{figure}[H]
    \centering
    \includegraphics[width=\linewidth]{Screenshot 2025-05-26 132203.png}
    \caption{Bắt đầu chương trình}
    \label{fig:bat_dau_chuong_trinh}
    \vspace{0.6cm}
    

    \includegraphics[width=\linewidth]{Screenshot 2025-05-26 132828.png}
    \caption{Kết thúc chương trình}
    \label{fig:ket_thuc_chuong_trinh}
    \vspace{0.6cm}
%======================================================

    \begin{subfigure}[b]{0.45\linewidth}
        \centering
        \includegraphics[width = 1.1\linewidth]{Screenshot 2025-05-26 132441.png}
        \caption{Trước hàm 'Nhap\_so'}
        \label{fig:truoc_nhap_so}
    \end{subfigure}
    \hfill
    \begin{subfigure}[b]{0.45\linewidth}
        \centering
        \includegraphics[width = 1.1\linewidth]{Screenshot 2025-05-26 132658.png}
        \caption{Sau hàm 'Nhap\_so'}\label{fig:sau_nhap_so}
    \end{subfigure}
    \caption{Trạng thái của hàm 'Nhap\_so'}
    \label{fig:ham_nhap_so}
    \vspace{0.6cm}

\end{figure}

%====================================================

\begin{figure}[H]
    \begin{subfigure}[b]{0.45\linewidth}
        \centering
        \includegraphics[width =1.1 \linewidth]{Screenshot 2025-05-26 132658.png}
        \caption{Trước hàm 'endl'}\label{fig:truoc_endl}
    \end{subfigure}
    \hfill
    \begin{subfigure}[b]{0.45\linewidth}
        \centering
        \includegraphics[width = 1.1\linewidth]{Screenshot 2025-05-26 132716.png}
        \caption{Sau hàm 'Nhap\_so'}\label{fig:sau_endl}
    \end{subfigure}
    \caption{Trạng thái của hàm 'endl'}
    \label{fig:ham_endl}

    \vspace{0.6cm}
    
    \centering
    \begin{subfigure}[b]{0.45\linewidth}
        \centering
        \includegraphics[width = 1.1\linewidth]{Screenshot 2025-05-26 132741.png}
        \caption{Trước hàm 'Dec\_to\_Bin'}\label{fig:truoc_dec_to_bin}
    \end{subfigure}
    \hfill
    \begin{subfigure}[b]{0.45\linewidth}
        \centering
        \includegraphics[width =1.1 \linewidth]{Screenshot 2025-05-26 132810.png}
        \caption{Sau hàm 'Dec\_to\_Bin'}\label{fig:sau_dec_to_bin}
    \end{subfigure}
    \caption{Trạng thái của hàm 'Dec\_to\_Bin'}
    \label{fig:ham_dec_to_bin}


\end{figure}

\begin{table}[H]
    \centering
    \caption{Giá trị các thanh ghi tại từng bước khảo sát}
    \begin{tabular}{|c|c|c|c|c|c|c|c|}
        \hline
        Bước & CS & IP & SS & SP & BP & DS & ES \\ \hline
        Bắt đầu chương trình & 0723 & 0000 & 0710 & 0100 & 0000 & 0700 & 0700 \\ \hline
        Trước hàm Nhap\_so & 0723 & \texttt{000C} & 0710 & 0100 & 0000 & 0720 & 0700 \\ \hline
        Sau hàm Nhap\_so & 0723 & \texttt{000F} & 0710 & 0100 & 0000 & 0720 & 0700 \\ \hline
        Trước hàm endl & 0723 & \texttt{000F} & 0710 & 0100 & 0000 & 0720 & 0700 \\ \hline
        Sau hàm endl & 0723 & 0012 & 0710 & 0100 & 0000 & 0720 & 0700 \\ \hline
        Trước hàm Dec\_to\_Bin & 0723 & \texttt{001B} & 0710 & 0100 & 0000 & 0720 & 0700 \\ \hline
        Sau hàm Dec\_to\_Bin & 0723 & \texttt{001E} & 0710 & 0100 & 0000 & 0720 & 0700 \\ \hline
        Kết thúc chương trình & \texttt{F400} & 0204 & 0710 & \texttt{00FA} & 0000 & 0720 & 0700 \\ \hline
    \end{tabular}
\end{table}

\subsubsection*{3. Giải thích nội dung các thanh ghi, trên cơ sở đó giải thích cơ chế quản lý bộ nhớ của hệ thống trong trường hợp cụ thể này}
\addcontentsline{toc}{subsubsection}{Giải thích nội dung các thanh ghi}
\textbf{\underline{Giải thích nội dung các thanh ghi}}
\begin{itemize}
    \item Thanh ghi phân đoạn - Quản lý vùng bộ nhớ
    \begin{itemize}
        \item CS (Code Segment): Trỏ đến vùng chứa mã lệnh chương trình (vùng .code).
        \item DS (Data Segment): Trỏ đến vùng chứa dữ liệu, như biến, chuỗi (vùng .data).
        \item SS (Stack Segment): Trỏ đến vùng stack, nơi lưu tạm thời dữ liệu khi push, pop.
        \item ES (Extra Segment): Vùng dữ liệu bổ sung, dùng khi làm việc với chuỗi.
    \end{itemize}
    \item Thanh ghi con trỏ – Quản lý vị trí trong segment
    \begin{itemize}
        \item IP (Instruction Pointer): Trỏ tới lệnh sẽ được thực thi tiếp theo. Tự động tăng sau mỗi lệnh.
        \item SP (Stack Pointer): Chỉ đến đỉnh stack hiện tại (offset) tính từ SS). Dùng trong push, pop.
        \item BP (Base Pointer): Dùng để truy cập dữ liệu trong stack, thường khi truyền tham số cho thủ tục.
    \end{itemize}
\end{itemize}

\textbf{\underline{Giải thích cơ chế bộ nhớ}}

\begin{itemize}
    
    \item Tổng quan về mô hình bộ nhớ (.model small):
    \begin{itemize}
        \item .model small: code và dữ liệu nằm trong cùng một segment (phân đoạn), không vượt quá 64KB.
        \item Các phân đoạn chính: 
        \begin{itemize}
            \item Code segment (CS): Chứa mã chương trình (main, Nhap\_so, Dec\_to\_Bin, endl)
            \item Data segment (DS): Chứa dữ liệu (x, y, tb1, tb2, crlf)
            \item Stack segment (SS): Chứa ngăn xếp (lưu tạm ax, địa chỉ trả về,...)
            \item Extra Segment	(ES): Thường dùng để hỗ trợ thao tác chuỗi/dữ liệu mở rộng
        \end{itemize}
    \end{itemize}
%================================================================
    \item CS (Code Segment) và IP (Instruction Pointer): 
    \begin{itemize}
        \item CS:IP kết hợp để xác định địa chỉ thực của lệnh đang được thực thi
        \item Khi chương trình bắt đầu: 
        \begin{itemize}
            \item CS trỏ đến đoạn mã lệnh (do hệ điều hành MS-DOS nạp).
            \item IP chứa offset đến lệnh hiện tại.
        \end{itemize}
        \item Mỗi khi CPU thực hiện xong một lệnh, nó cập nhật IP để trỏ đến lệnh tiếp theo
        \item Với những lệnh thông thường như MOV, ADD, SUB,... thì IP sẽ trỏ đến những câu lệnh một cách tuần tự từ trên xuống
        \item Với những lệnh như JMP, CALL, RET thì chương trình không còn chạy tuần tự, mà sẽ nhảy đến nơi khác theo chỉ định.
        \begin{itemize}
            \item Lệnh JMP: Sau khi thực hiện 'JMP lap1' thì IP lại trỏ đến lệnh sau label (nhãn) lap1 là lệnh MOV AH,1
            \begin{figure}[H]
                \begin{subfigure}[b]{0.4\linewidth}
                    \centering
                    \includegraphics[width = \linewidth]{IP khi den lenh JMP.png}
                    \caption{IP khi đến lệnh JMP}
                \end{subfigure}
                \hfill
                \begin{subfigure}[b]{0.4\linewidth}
                    \centering
                    \includegraphics[width = \linewidth]{IP sau thuc hien JMP.png}
                    \caption{IP sau thực hiện lệnh JMP}
                \end{subfigure}
                \caption{Lệnh JMP}
            \end{figure}
            \item Lệnh CALL và RET: Khi IP chạy đến lệnh 'CALL Nhap\_so' thì CPU sẽ đẩy địa chỉ của lệnh sau 'CALL Nhap\_so' là 'CALL endl' vào Stack và sau khi kết thúc lệnh RET của 'Nhap\_so' thì địa chỉ của lệnh 'CALL endl' được lấy ra và IP trỏ tới để chương trình tiếp tục chạy
            \begin{figure}[H]
                \begin{subfigure}[b]{0.4\linewidth}
                    \centering
                    \includegraphics[width = \linewidth]{dia chi cua ham endl duoc day vao stack.png}
                    \caption{IP nhảy vào Nhap\_so}
                \end{subfigure}
                \hfill
                \begin{subfigure}[b]{0.4\linewidth}
                    \centering
                    \includegraphics[width = \linewidth]{dia chi cua ham endl sau ham Nhap_so.png}
                    \caption{IP sau khi kết thúc Nhap\_so}
                \end{subfigure}
                \caption{Lệnh CALL và RET}
            \end{figure}
            
        \end{itemize}
    \end{itemize}

%==================================================================
    \item DS (Data Segment): 
    \begin{itemize}
        \item Dùng để truy xuất biến trong (.data) như: x, y, tb1, tb2, crlf.
        \item Được khởi tạo tại dòng:
        \begin{figure}[h]
            \centering
            \includegraphics[width=0.5\linewidth]{khởi tạo ds.png}
            \caption{Khởi tạo DS}
        \end{figure}

        $\rightarrow$ @data trả về địa chỉ segment chứa dữ liệu thông qua thanh ghi AX trunng gian đưa vào thanh ghi DS.

            $\Rightarrow$ Quản lí các biến x , y, tb1, tb2, crlf.
    \end{itemize}

%==================================================================
    \item SS (Stack Segment) và SP (Stack Pointer):
    \begin{itemize}
        \item SS và SP dùng để truy xuất ngăn xếp.
        \item Stack được khai báo qua 9 (.stack 100h ) $\rightarrow$ cấp 256 byte cho stack.
        \item Hệ thống sẽ tự động khởi tạo SS và SP khi chương trình bắt đầu.
        \begin{itemize}
            \item SS trỏ vào segment stack.
            \item SP thường khởi tạo ở đỉnh stack (offset 0100h).
            \begin{figure}[H]
            \centering
            \includegraphics[width=0.8\textwidth]{dia chi khoi tao stack.png}
            \caption{Địa chỉ SS và SP khi chương trình chạy}
        \end{figure}
        \end{itemize}
        \item Ví dụ khi hàm chạy đến hàm 'Nhap\_so'
        \begin{figure}[H]
            \centering
            \includegraphics[width=0.8\linewidth]{stack truoc khi vao ham Nhap_so.png}
            \caption{Stack trước hàm Nhap\_so}
        \end{figure}
        \item Địa chỉ 0710:00FE có giá trị là 0202h
        \begin{figure}[H]
            \centering
            \includegraphics[width=0.8\linewidth]{dia chi cua ham endl duoc day vao stack.png}
            \caption{Stack trong hàm Nhap\_so}
        \end{figure}
        \item Địa chỉ 0710:00FE đã thay đổi giá trị thành 000F $\Rightarrow$ CPU đẩy địa chỉ của lệnh kế tiếp (call endl)  vào Stack sau khi thực hiện hàm 'Nhap\_so'.
        \begin{figure}[H]
            \centering
            \includegraphics[width=0.8\linewidth]{dia chi cua ham endl sau ham Nhap_so.png}
            \caption{Stack sau hàm Nhap\_so}
        \end{figure}
        \item Sau hàm Nhap\_so thì giá trị 000F của 0710:00FE được pop ra khỏi stack và CS:IP = 0723:0022 $\Rightarrow$ IP trỏ tới địa chỉ lệnh 'call endl' 
    \end{itemize}

\item BP (Base Pointer):
\begin{itemize}
    \item BP thường dùng trong thao tác với stack (ví dụ như khi xử lý tham số trong thủ tục).
    \item Tuy nhiên không được sử dụng trong bài 'Chuyển số thập phân thành nhị phân', nên không ảnh hưởng đến quản lý bộ nhớ trong trường hợp này.
\end{itemize}

%==============================================
\item  ES (Extra Segment):
\begin{itemize}
    \item ES là thanh ghi segment bổ sung, thường dùng để thao tác với các chuỗi hoặc vùng nhớ đặc biệt.
    \item Trong chương trình này không sử dụng ES, nên không có quản lý bộ nhớ liên quan đến nó.
\end{itemize}

\end{itemize}